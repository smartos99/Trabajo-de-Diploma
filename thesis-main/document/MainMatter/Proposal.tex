\chapter{Propuesta}\label{chapter:proposal}

Con el objetivo de lograr una síntesis de voz satisfactoria y después de realizar un estudio de las tecnologías que se utilizan para este fin, se elige Coqui TTS como herramienta base. Coqui cuenta con una gran variedad de modelos preentrenados en más de 20 idiomas.  
El primer paso en el desarrollo del sintetizador fue instalar la biblioteca Coqui TTS, de acuerdo a las indicaciones del repositorio oficial[\cite{coqui-doc}]. Se realizó un estudio del comportamiento de combinaciones de parejas modelo TTS y VoCoder, para evaluar cuáles ofrecían mejores resultados:



\begin{figure}[H]
	\centering
	\includegraphics[width=0.3\linewidth]{Graphics/en_ljspeech}
	\caption{Modelo TTS Tacotron2-DDC entrenado en Inglés}
	\label{enljspeech}
\end{figure}

\begin{figure}[H]
	\centering
	\includegraphics[width=0.3\linewidth]{Graphics/fastpitch}
	\caption{Modelo TTS Fast-Pitch entrenado en Inglés}
	\label{fastpitch}
\end{figure}
\begin{figure}[H]
	\centering
	\includegraphics[width=0.4\linewidth]{Graphics/glow_tts}
	\caption{Modelo TTS Glow-TTS entrenado en Inglés}
	\label{glowtts}
\end{figure}

\begin{figure}[H]
	\centering
	\includegraphics[width=0.7\linewidth]{Graphics/es_mai}
	\caption{Modelo TTS Tacotron2-DDC entrenado en Español}
	\label{esmai}
\end{figure}


Los modelos \textbf{Tacotron2-DDC, Fast-pitch y Glow-tts} preentrenados en idioma inglés al combinarse con VoCoders, como se observa en las figuras \ref{enljspeech}, \ref{fastpitch}, \ref{glowtts} para un texto en español arroja distintos resultados, en algunos casos solo producen ruido mientras que los más satisfactorios, producen un discurso con una pronunciación propia de una persona de habla inglesa hablando español.

Por otro lado Coqui solo cuenta con un modelo preentrenado en español, el \textbf{Tacontron-DDC}, que está entrenado sobre la base de datos de M-AILABS[\cite{mailabs}]; este modelo fue probado junto a los VoCoders preentrenados como se muestra en la figura \ref{esmai}. Los mejores resultados fueron arrojados por las combinaciones de \textbf{Tacontron-DDC} con los modelos: Univnet entrenado sobre la base de datos Ljspeech en inglés, y Wavegrad entrenado sobre el conjunto de datos LibriTTS, aunque ambos presentan problemáticas como la mala pronunciación de la letra ñ.\\

De acuerdo con la documentación de Coqui, y con el objetivo de la investigación que consiste en desarrollar un sintetizador de voz en español con voz cubana, se proponen tres enfoques:

\begin{enumerate}
	\item Realizar un \textit{fine-tuning} o reentrenamiento, utilizando una base de datos con voces cubanas, al modelo preentrenado de Coqui[\cite{coqui-doc}], \textbf{Tacontron-DDC}, que es el único disponible en español. 
	
	\item Realizar el entrenamiendo desde cero de un modelo, en este caso se podría entrenar cualquier modelo disponible. \textbf{VITS} es un buen candidato pues en general produce resultados bastante satisfactorios.
	
	\item Realizar un \textit{fine-tuning} a modelos VITS preentrenados en idiomas distintos al español.
	
	
	
\end{enumerate}

Y a partir de los anteriores evaluar y comparar resultados.

Para el desarrollo de cualquiera de los anteriores es imprescindible la construcción de un conjunto de datos que se adapten al modelo y a las necesidades de la investigación.


\section{Creación de la base de datos}

Para entrenar un modelo TTS, se necesita un conjunto de datos con grabaciones de voz y transcripciones, en este caso fue una base de datos en español con voces cubanas. El discurso debe dividirse en clips de audio y cada clip necesita su transcripción correspondiente. La base de datos debe poseer una organización específica, que se eligirá luego, de forma tal que el cargador de datos de Coqui TTS sea capaz de cargarlos para utilizar en el entrenamiento[\cite{formatting-dataset}].    

\subsubsection{Formato wav}
(***)



\subsubsection{Hablar del idioman español}
(***)\\


La base de datos conformanda debe tener una buena cobertura del idioma en el que se desea entrenar el modelo. Debe cubrir la variedad fonémica, los sonidos excepcionales y las sílabas. 

\subsection*{Fine-Tuning}
Entrenar correctamente un modelo de aprendizaje profundo requiere generalmente de una gran base de datos y de un extenso entrenamiento.

Si se dispone del material necesario y del tiempo para entrenar el algoritmo, estos requisitos no suponen ningún problema, pero, si la base de datos es pequeña o el modelo no se entrena lo suficiente, el aprendizaje podría no ser completo.

El \textit{fine-tuning} es una forma de transferencia de aprendizaje, consiste en aprovechar la información que se ha aprendido de
la resolución de un problema y utilizarla sobre otro distinto, pero que comparte ciertas características. Se usan los conocimientos adquiridos por una red convolucional para transmitirlos a una nueva red convolucional. Esta nueva red convolucional que se crea no tiene por qué modificar la red original y puede simplemente aprender de ella, sin embargo también es válido el caso donde no solo se modifica la red original, sino que se vuelve a entrenar para aprender más conceptos.

En la presente investigación, se utiliza el reentrenamiento, para a partir modelo previamente entrenado y realizar un nuevo entrenamientor para mejorar su rendimiento en un conjunto de datos diferente.\\

\section{Fine-Tuning de Tacontron-DDC}

Se selecciona como primera variante el reentrenamiento(\textit{fine-tuning} en inglés) del modelo \textbf{Tacontron-DDC} en español, pues plantea ventajas tales como un aprendizaje más rápido, ya que un modelo preentrenado ya tiene aprendidas funcionalidades que son relevantes para la tarea de producir un discurso. Además convergerá más rápido en el nuevo \textit{dataset}, lo que reducirá el costo de entrenamiento y permitirá el proceso de experimentación más rápidamente. Y de acuerdo a la teoría se pueden obtener buenos resultados con un conjunto de datos más pequeño.\\

Luego de tener entrenado el modelo acústico(TTS) con una base de datos construida con voces cubanas, es posible que alguno de los VoCoders preentrenados disponibles produzca una salida con las características deseadas, en caso contrario se debería entrenar el VoCoder con los datos del \textit{dataset} construido.


El proceso consiste en modificar la configuración original del modelo preentrenado seleccionado, es decir, especificar la base de datos a utilizar en el reentrenamiento, los detalles acústicos que reflejen las características del nuevo conjunto de datos, el nombre del nuevo modelo ajustado, la dirección donde se guardará el modelo reentrenado, entre otras cuestiones. Para la mayoría de los parámetros se tomaron las características del modelo original \textbf{Tacontron-DDC} en español.



\section{Entrenamiento de modelo VITS desde cero}

Existen varios modelos de texto a voz de extremo a extremo que permiten el entrenamiento en una sola etapa y el muestreo en paralelo, sin embargo, generalmente la calidad de la muestra no coincide con la de los sistemas TTS de dos etapas. 

Se selecciona VITS porque es un método TTS paralelo de extremo a extremo que genera un sonido de audio más natural que los modelos actuales de dos etapas. Y de acuerdo a una evaluación humana subjetiva (puntuación de opinión media, o MOS), muestra que el modelo supera a los mejores sistemas TTS disponibles públicamente y logra un MOS comparable a la realidad del terreno.

Con este enfoque se debe entrenar la red neuronal de VITS partiendo de cero. Una desventaja es que cae una gran responsabilidad sobre el conjunto de datos de entrenamiento, pues debe tener una gran riqueza y muchos clips de audio.
	
	
\section{Fine-tuning de modelos VITS preentrenados en otros idiomas}

\subsection{Italiano}
Las culturas hispánicas tienen mucho en común con la cultura italiana, sobre todo en lo que concierne al lenguaje y la comunicación. Tanto el español como el italiano son lenguas romances, derivadas del latín, siendo justamente de las más similares, incluso más que el francés o el rumano.

Es por este hecho que el italiano y el español comparten palabras muy parecidas y siguen la misma estructura gramatical. Entre ellos existe un grado de similitud léxica del 82$\%$, lo que además indica que son idiomas fáciles de aprender para sus respectivos hablantes. Sin embargo, las características que tienen en común van más allá de su origen y de la gramática.

Una similitud entre ambas lenguas es la cantidad de vocales en el alfabeto, aunque en italiano las vocales tienden a matizarse con sonidos más abiertos y con un acento grave. Es muy común que el parecido existente entre las palabras en italiano y español se vea afectada sólo por una o más vocales, como por ejemplo: vecino y vicino, cámara y camera, igual y uguale. Por supuesto existen un gran número de diferencias, aunque son más notables en vocabulario que en lo que sería la pronunciación. 

Debido a todo esto un modelo entrenado en italiano sería el candidato ideal para realizar un ajuste sobre una base de datos en español. 
\subsection{Inglés}

A pesar de las diferencias evidentes, en realidad el español y el inglés se parecen más de lo se cree. La primera de las semejanzas entre el inglés y el español es que usan el alfabeto romano, lo cual provoca que los sonidos de ambos idiomas sean similares, aunque no plenamente iguales. El mejor ejemplo lo constituyen las vocales: mientras en el español sólo se reconocen 5 sonidos, uno para cada vocal, en el inglés encontramos más de 14 sonidos, pues hay vocales cortas y largas, las cuales se catalogan así por la duración de su pronunciación. Estos dos idiomas son alfabéticos, de modo que a través de letras se pueden representar sus sonidos; el español comparte 2/3 de sus fonemas con el inglés.

La estructura gramatical del inglés es similar a la del español en más de un 90$\%$, pero la del inglés es muchísimo más simple. Además, el vocabulario se forma con prefijos y sufijos tanto en inglés como en español (que son en su mayoría de raíz latina en ambos idiomas).

%Tea (i)         Esta vocal es más aguda que la “i”.
%Fork (o)       Parecido a la “o” pero más redondeada y profunda puede sonar un poco similar a la “u”.
%Cook (u)       Similar a la “u” pero sin redondear los labios


Las consonantesv, ll, h, j, r, rr, z, y  la x son pronunciadas muy diferentes en español y en inglés. La consonante ñ no existe en inglés; el sonido que se conoce de esta consonante en español se escribe en ingles ny. Las combinaciones de algunas consonantes con vocales cambia totalmente el sentido de la pronunciación. Por ejemplo: la combinación de la “q” y la “u” en las palabras como queen, quiet o quick la pronunciación de  la “u” en inglés se percibe.

Teniendo en cuenta la similitud, y al mismo tiempo las marcadas diferencias entre ambos idiomas, se escoge un modelo VITS preentrenado en inglés para realizar un \textit{fine-tuning} con la base de datos cubana.
